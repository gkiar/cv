%%%%%%%%%%%%%%%%%%%%%%%%%%%%%%%%%%%%%%%%%
% Friggeri Resume/CV
% XeLaTeX Template
% Version 1.2 (3/5/15)
%
% This template has been downloaded from:
% http://www.LaTeXTemplates.com
%
% Original author:
% Adrien Friggeri (adrien@friggeri.net)
% https://github.com/afriggeri/CV
%
% License:
% CC BY-NC-SA 3.0 (http://creativecommons.org/licenses/by-nc-sa/3.0/)
%
% Important notes:
% This template needs to be compiled with XeLaTeX and the bibliography, if used,
% needs to be compiled with biber rather than bibtex.
%
%%%%%%%%%%%%%%%%%%%%%%%%%%%%%%%%%%%%%%%%%

\documentclass[]{friggeri-cv} % Add 'print' as an option into the square bracket to remove colors from this template for printing
\addbibresource{gkiarcv.bib} % Specify the bibliography file to include publications
\geometry{left=2cm,right=2cm}

\begin{document}

\header{gregory}{kiar}{Research Scientist, Child Mind Institute}
{{\color{red} \faEnvelope[regular]}~\href{mailto:gregory.kiar@childmind.org}{gregory.kiar@childmind.org}\quad|\quad {\color{brown}\faMobile*}~(646)~880-3737\quad | \quad {\color{lightgray} \faGlobe}~\href{https://gkiar.me}{gkiar.me} \\ {\color{purple}\faGithub}~\href{https://github.com/gkiar}{gkiar}\quad|\quad{\color{orcidgreen}\faOrcid}~\href{http://orcid.org/0000-0001-8915-496X}{0000-0001-8915-496X}\quad|\quad {\color{blue}\faTwitter}~\href{https://twitter.com/g_kiar}{g\_kiar}} % Your name and current job title/field

% \newgeometry{left=2cm, bottom=2.5cm, right=1.5cm}
\renewcommand{\bfseries}{\headingfont\color{headercolor}}
\renewcommand{\entry}[4]{%
  #1&\parbox[t]{15.7cm}{%
    \textbf{#2}%
    \hfill%
    {\footnotesize\addfontfeature{Color=lightgray} #3}\\%
    #4\vspace{\parsep}%
  }\\}

% add this below the last entry on your first page
% \newgeometry{left=2cm, bottom=2.5cm, right=1.5cm}
\renewcommand{\bfseries}{\headingfont\color{headercolor}}
\renewcommand{\entry}[4]{%
  #1&\parbox[t]{15.7cm}{%
    \textbf{#2}%
    \hfill%
    {\footnotesize\addfontfeature{Color=lightgray} #3}\\%
    #4\vspace{\parsep}%
  }\\}

% add this below the last entry on your first page
% \newgeometry{left=2cm, bottom=2.5cm, right=1.5cm}
\renewcommand{\bfseries}{\headingfont\color{headercolor}}
\renewcommand{\entry}[4]{%
  #1&\parbox[t]{15.7cm}{%
    \textbf{#2}%
    \hfill%
    {\footnotesize\addfontfeature{Color=lightgray} #3}\\%
    #4\vspace{\parsep}%
  }\\}

% add this below the last entry on your first page
% \input{midamble.tex}



%----------------------------------------------------------------------------------------
%	SIDEBAR SECTION
%----------------------------------------------------------------------------------------
% Uncomment  the aside and move down the midamble if I'm applying for a job
%

% \begin{aside} % In the aside, each new line forces a line break
% %\includegraphics[width=\textwidth]{./headshot.png}
% \subsection{languages}
% native English speaker
% \subsection{programming}
% Python, R, Bash {\color{red} $\varheartsuit$}
% MATLAB, C++, AWS,
% Ruby, LaTeX, etc.
% \subsection{soft skills}
% writing, teaching, leadership, design, problem~solving
% \subsection{research interests}
% uncertainty, reproducibility, neuroscience, machine~learning
% \end{aside}

%----------------------------------------------------------------------------------------
%	EDUCATION SECTION
%----------------------------------------------------------------------------------------

\section{education}

\begin{entrylist}

%------------------------------------------------

\entry
{2017 -- 2021}
{Ph.D. {\normalfont in Biomedical Engineering}}
{McGill University, Montreal, QC}
{Thesis work supervised by Alan Evans and Tristan Glatard on a project entitled:\\ This is Your Brain on Disk: The
Impact of Numerical Instabilities in Neuroscience. Project involved the development of high performance computing
infrastructures, the instrumentation and perturbation of neuroimaging pipelines, the evaluation of these perturbations
in an analytic context, and the application of perturbed derivatives towards data augmentation in machine learning
applications. All code and data have been made publicly available.}

%------------------------------------------------

\entry
{2014 -- 2016}
{M.S.E {\normalfont in Biomedical Engineering}}
{Johns Hopkins University, Baltimore, MD}
{Thesis work was supervised by Joshua T. Vogelstein on a project entitled:\\GREMLIN:
Graph Estimation from MR images Leading to Inference in Neuroscience. All code and data have
been made publicly available.}

%------------------------------------------------

\entry
{2010 -- 2014}
{B.Eng {\normalfont in Biomedical and Electrical Engineering}}
{Carleton University, Ottawa, ON}
{Capstone work was supervised by Leonard MacEachern on a project entitled:\\Electrical
muscle stimulation with concurrent EMG feedback of the upper arm for applications in stroke
rehabilitation.}

%------------------------------------------------

\entry
{2018}
{Software and Data Carpentry Instructor Training}
{Compute Canada, Toronto, ON}
{Running workshops in the context of an evidence-based instructional pedagogy.}

%------------------------------------------------

\entry
{2016}
{Exploring the Human Connectome}
{The Human Connectome Project, Boston, MA}
{Development and deployment of connectome estimation pipelines.}

%------------------------------------------------

\entry
{2015}
{Presenting Data and Information}
{Edward Tufte, Baltimore, MD}
{Cultivate skills in effective communication with scientific figures.}

\end{entrylist}

%----------------------------------------------------------------------------------------
%	WORK EXPERIENCE SECTION
%----------------------------------------------------------------------------------------

\section{experience}

\subsection{Research Experience}

\begin{entrylist}
\entry
{04/21 -- now}
{Child Mind Institute}
{New York City, NY}
{\job{Research Scientist} \\
Responsible for evaluating and improving the trustworthiness of tools and techniques used to study the brain.
Evaluates the numerical stability of analyses to inform decision making surrounding robust data collection, image
processing, and ultimately biomarker discovery. Applies expertise in computational statistics, uncertainty
quantification, and machine learning towards developing robust and generalizable models of the
healthy and diseased brain.}

\entry
{05/17 -- 04/21}
{McGill Centre for Integrative Neuroscience}
{Montreal, QC}
{\job{Software Developer \& Researcher} \\
Responsible for the exploration and integration of distributed software software services with high
performance computing clouds and clusters, providing development, training, and support towards the
use of tools and services within international collaborations. Focused on the development of methods
for evaluating the trustworthiness and stability of neuroimaging tools and experiments.}

\entry
{04/19 -- 07/19}
{Empenn, Inria Rennes - Bretagne Atlantique}
{Rennes, France}
{\job{Research Intern}\\
Developed web crawler to scrape public neuroimaging databases for processed functional activation maps. Constructed
workflow for metadata based QC at scale with 10,000s of samples. Designed and trained a convolutional neural network
model for the identification of consensus activation maps across populations.}

\end{entrylist}

%\newgeometry{left=2cm, bottom=2.5cm, right=1.5cm}
\renewcommand{\bfseries}{\headingfont\color{headercolor}}
\renewcommand{\entry}[4]{%
  #1&\parbox[t]{15.7cm}{%
    \textbf{#2}%
    \hfill%
    {\footnotesize\addfontfeature{Color=lightgray} #3}\\%
    #4\vspace{\parsep}%
  }\\}

% add this below the last entry on your first page
% \newgeometry{left=2cm, bottom=2.5cm, right=1.5cm}
\renewcommand{\bfseries}{\headingfont\color{headercolor}}
\renewcommand{\entry}[4]{%
  #1&\parbox[t]{15.7cm}{%
    \textbf{#2}%
    \hfill%
    {\footnotesize\addfontfeature{Color=lightgray} #3}\\%
    #4\vspace{\parsep}%
  }\\}

% add this below the last entry on your first page
% \newgeometry{left=2cm, bottom=2.5cm, right=1.5cm}
\renewcommand{\bfseries}{\headingfont\color{headercolor}}
\renewcommand{\entry}[4]{%
  #1&\parbox[t]{15.7cm}{%
    \textbf{#2}%
    \hfill%
    {\footnotesize\addfontfeature{Color=lightgray} #3}\\%
    #4\vspace{\parsep}%
  }\\}

% add this below the last entry on your first page
% \input{midamble.tex}




\begin{entrylist}
\entry
{09/14 -- 05/17}
{Center for Imaging Science, Johns Hopkins University}
{Baltimore, MD}
{\job{Research Engineer}\\
Development and maintenance of an open-source pipeline for structural connectome estimation in humans
and implemented statistical algorithms for quality control of data derivatives. Publicly released data
products to lower the barrier to entry for neuroscience research. Chiefly responsible for grant reporting
and public presence at conferences and workshops.}

\entry
{06/13 -- 09/13}
{Dept. of Systems and Computer Engineering, Carleton University}
{Ottawa, ON}
{\job{Research Assistant with Dr. Rafik Goubran}\\
Developed wireless medical data publish-subscribe system for viewing patient vital signs remotely.}

\entry
{06/12 -- 09/12}
{Dept. of Systems and Computer Engineering, Carleton University}
{Ottawa, ON}
{\job{Research Assistant with Dr. Andy Adler}\\
Utilized neural networks for inverse modeling of real and simulated biological systems.}

\entry
{06/11 -- 09/11}
{Dept. of Biology, Carleton University}
{Ottawa, ON}
{\job{Research Assistant with Dr. Jeffrey Dawson}\\
Developed robotics platform for studying insect locomotion patterns and behaviour.}

\entry
{01/09 -- 09/09}
{CRC, Ottawa Hospital Research Institute}
{Ottawa, ON}
{\job{Research Assistant with Dr. Jim Dimitroulakos}\\
Tested combination therapies of Lovastatin and Cisplatin drugs on colon and breast cancer strains.}
\end{entrylist}

\subsection{Teaching Experience}

\begin{entrylist}
\entry
{01/19 -- 01/20}
{Concordia Continuing Education}
{Montreal, QC}
{\job{Instructor \& Curriculum Developer} \\
Responsible for the training of working professionals in the basics of "Big Data Technology," including fundamental
tools for software development such as Unix, Git, and Docker, and software for numerical analysis such as Python and R.
Core contributor in the development of new courses within the "Big Data Solutions for Business" diploma program.}

\entry
{05/17 -- now}
{McGill University, OHBM, Brainhack School, Brain Intensive, others}
{Montreal, QC}
{\job{Neuroinformatics Instructor} \\
Regularly plan and teach a series of workshop introducing neuroscientists and trainees to methods in neuroinformatics.
Developed and publicly released all course content on GitHub under the "Brainhack101" moniker and several videos on
YouTube under the "BrainIntensive" profile.}

\entry
{09/14 -- 05/17}
{Dept. of Biomedical Engineering, Johns Hopkins University}
{Baltimore, MD}
{\job{Teaching Assistant} \\
Responsible for instruction, evaluation, and content design for: Freshman Modeling and Design
for BME (2014, 2015), Systems and Controls (2015), Statistical Connectomics (2015), The Art of
Data Science (2016), NeuroData Design (2016). Spent more than 500 hours working with students.}

\entry
{01/\{15, 16, 17\}}
{Dept. of Computer Science, Johns Hopkins University}
{Baltimore, MD}
{\job{Instructor}\\
Responsible for instruction, evaluation, and content design for intensive 3-week project-based course on an
introduction to connectomics research across multiple scales and experimental modalities. Spent more than 300 hours
planning, designing course content, and working with students.}

\entry
{09/12 -- 05/14}
{Student Academic Success Center, Carleton University}
{Ottawa, ON}
{\job{Facilitator for Peer-Assisted Study Sessions}\\
Instructed and demonstrated mastery of principles in electromagnetism and power engineering. Spent more than 300 hours
working with students.}

\entry
{08/13 -- 05/14}
{Student Academic Success Center, Carleton University}
{Ottawa, ON}
{\job{Facilitator Team Leader}\\
Provided training, mentoring, and coaching to student instructors in a variety of disciplines. Spent more than 100
hours training and working with facilitators.}

\end{entrylist}

\begin{entrylist}
\entry
{01/13 -- 06/14}
{Dept. of Systems and Computer Engineering, Carleton University}
{Ottawa, ON}
{\job{Teaching Assistant}\\
Instructed introductory level C++ programming. Led lab sessions and instructional workshops. Spent more than 300 hours
working with students.}
\end{entrylist}


%----------------------------------------------------------------------------------------
%	AWARDS SECTION
%----------------------------------------------------------------------------------------
\section{grants \& awards}
\subsection{grants}
\begin{entrylist}
\entry
{2022 -- 2025}
{NIH NIMH, 1RF1MH130859 | PI: Gregory Kiar}
{Awarded Amount: \$1,504,004.00}
{Improving the robustness of neuroimaging through exploitation of variability in processing pipelines}

\entry
{2022 -- 2023}
{NSF XSEDE, MED220009 | PI: Gregory Kiar}
{Awarded Amount: \$117,077.80}
{Preprocessing and sharing of large-scale open neuroimaging datasets}

\entry
{2022 -- 2023}
{NSF XSEDE, BIO220056 | Role: Co-Investigator | PI: Ting Xu}
{Awarded Amount: \$30,786.95}
{Mapping Large-scale Brain Development between Human and Nonhuman Primate}

\entry
{2021 -- 2023}
{Michael J. Fox Foundation | Role: Consultant | PI: Tristan Glatard}
{Awarded Amount: \$305,254.00}
{Improving the generalizability and robustness of MRI-derived biomarker of \\ Parkinson's Disease through analytical and data variability evaluations}

\entry
{2021 -- 2022}
{NSF XSEDE, CIS210056 | PI: Gregory Kiar}
{Awarded Amount: \$880.00}
{Application of uncertainty quantification for neuroimaging software design,\\ testing, and analysis}

\entry
{2018 -- 2021}
{NSERC Canada, CGSD3-519497-2018 | PI: Gregory Kiar}
{Awarded Amount: \$105,000.00}
{Supporting scalable computing in neuroimaging for the exploration of \\numerical instabilities and their impact}
\end{entrylist}

\subsection{awards}
\begin{entrylist}
\vspace{-7pt}

\entry
{2020}
{Research Scholar Award}
{Canadian Open Neuroscience Platform, Montreal, QC}
{}
\vspace{-7pt}
% $25000
% For work on studying instabilities

\entry
{2019}
{Young Investigator Award}
{Sage Bionetworks, Seattle, WA}
{}
\vspace{-7pt}
% $1000
% For being a young human who does open science things

\entry
{2019}
{Instructor Training Fellowship}
{Repronim, Worcester, MA}
{}
\vspace{-7pt}
% $1000
% For training to be a repronim instructor

\entry
{2019}
{Globalink Research Award}
{Mitacs, Montreal, QC}
{}
\vspace{-7pt}
% $6000
% For work at INRIA on computational stability

\entry
{2018}
{Michael Smith Foreign Study Supplement}
{NSERC, Ottawa, ON}
{}
\vspace{-7pt}
% $5200
% For work at INRIA on computational stability

\entry
{2017}
{Healthy Brains for Healthy Lives Doctoral Fellowship}
{McGill University, Montreal, QC}
{}
\vspace{-7pt}
% $15000

\entry
{2017}
{CRN Coding Sprint Project Award}
{Stanford University, Palo Alto, CA}
{}
\vspace{-7pt}
% $2000

\entry
{2017}
{OHBM BrainHack Travel Award}
{OHBM, Minneapolis, MN}
{}
\vspace{-7pt}
% $500

\entry
{2014 -- 2016}
{Full-tuition Master's Degree Fellowship}
{Johns Hopkins University, Baltimore, MD}
{}
\vspace{-7pt}
% $100,000

\entry
{2014}
{Graduated with Distinction}
{Carleton University, Ottawa, ON}
{}
\vspace{-7pt}

\entry
{2014}
{Greatest Social Impact Paper}
{Professional Engineering Ontario (PEO), Ottawa, ON}
{}
\vspace{-7pt}
% $500
%{Awarded to the capstone project with the potential to produce the largest positive societal impact.}

\entry
{2014}
{SEED Fund}
{Carleton University Engineering Alumni, Ottawa, ON}
{}
\vspace{-7pt}
% $800
%{Awarded to the capstone project deemed most likely to become a successful startup.}

\entry
{2014}
{IEEE Papers Showcase Local Winner}
{IEEE Ottawa-Carleton Chapter, Ottawa, ON}
{}
\vspace{-7pt}
% $500
%{Awarded to the capstone project best demonstrating mastery of core electrical engineering principles.}

\entry
{2014}
{Carleton Electronics Project Competition Champion}
{Carleton University, Ottawa, ON}
{}
\vspace{-7pt}
%{Awarded to the capstone project best demonstrating mastery of core electrical engineering principles.}

\entry
{2013}
{Engineering '65 and '66 Scholarship}
{Carleton University, Ottawa, ON}
{}
\vspace{-7pt}
% $2,000
%{Awarded to students maintaining a GPA above a 10/12 (the equivalent of an A).}

\entry
{2012}
{Clarence C. Gibson Scholarship}
{Carleton University, Ottawa, ON}
{}
\vspace{-7pt}
% $2,000
%{Awarded to students maintaining a GPA above a 10/12 (the equivalent of an A).}
\end{entrylist}

\section{supervision \& academic mentorship}
\begin{enumerate}
\item Maya Roberts (Research Assistant, Child Mind Institute; 2022)
\item Amy Gutierrez (Software Developer, Child Mind Institute; with Michael P. Milham; 2022)
\item Teresa George (Software Developer, Child Mind Institute; with Michael P. Milham; 2022)
\item Jon Clucas (Software Developer, Child Mind Institute; with Michael P. Milham; 2022)
\item Xinhui Li (Software Developer, Child Mind Institute; with Michael P. Milham; 2021)
\item Ali Salari (PhD in Computer Science, Concordia University; with Tristan Glatard; 2019–2022)
\item Hamidreza Heidarzadeh (MSc in Computer Science, Concordia University; with Tristan Glatard; 2018–2019)
\end{enumerate}

%----------------------------------------------------------------------------------------
%	EXTRACURRICULARS SECTION
%----------------------------------------------------------------------------------------
\section{memberships \& extracurriculars}

\begin{entrylist}
\entry
{2020 -- now}
{XSEDE, NSF}
{Alexandria, VA}
{XSEDE Review Allocation Committee Member}

\entry
{2017 -- now}
{Various Neuroinformatics-based Hackathons and Courses}
{Montreal, QC}
{Hackathon Chair, Organizer, \& Instructor}

\entry
{2020}
{COVID-19 HPC Consortium}
{Global}
{Review Allocation Committee Member}

\entry
{2017 -- 2020}
{Canadian Open Neuroscience Platform Training Committee}
{Montreal, QC}
{Trainee Representative}

\entry
{2017 -- 2020}
{OHBM Open Science SIG}
{Minneapolis, MN}
{Treasurer, Educational Committee Liaison}

\entry
{2018 -- 2019}
{Ludmer Centre Seeds of Change Campaign}
{Montreal, QC}
{Trainee Ambassador}

\entry
{2017 -- 2018}
{OHBM Open Science SIG}
{Minneapolis, MN}
{Hackathon Chair}

\entry
{2017 -- 2018}
{Healthy Brains for Healthy Lives Trainee Committee}
{Montreal, QC}
{President (Neuroinformatics)}

\entry
{2015 -- 2017}
{College Prep Program}
{Baltimore, MD}
{College Mentor, SAT Coach, \& Essay Reviewer}

\entry
{2014 -- 2016}
{Thread}
{Baltimore, MD}
{Volunteer supervisor \& student mentor}

\entry
{2013 -- 2014}
{Carleton University Biomedical Engineering Society}
{Ottawa, ON}
{President}

\entry
{2010 -- 2011}
{Carleton University Student Emergency Response Team}
{Ottawa, ON}
{Emergency First Responder}
\end{entrylist}

%----------------------------------------------------------------------------------------
%	INTERESTS SECTION
%----------------------------------------------------------------------------------------

\section{reviewed for}
\begin{enumerate}
\item COVID-19 High Performance Computing Consortium
\item Cluster Computing
\item Extreme Science and Engineering Discovery Environment (XSEDE)
\item Frontiers in Neuroinformatics
\item Gigascience
\item Journal of Open Source Software
\item Medical Image Analysis
\item Nature Communications Biology
\item Neuroimage
\item Practice \& Experience in Advanced Research Computing Conference (PEARCC)
\item Scipy Conference
\end{enumerate}

%----------------------------------------------------------------------------------------
%	PUBLICATIONS SECTION
%----------------------------------------------------------------------------------------
\section{publications}
% \printbibsection{report}{under review} % Print all articles from the bibliography

\printbibsection{manual}{pre-prints} % Print all articles from the bibliography

\printbibsection{article}{articles in peer-reviewed journals} % Print all articles from the bibliography

\printbibsection{inproceedings}{proceedings in international peer-reviewed conferences} % Print all inproceedings entries from the bibliography

\printbibsection{inbook}{book chapters} % Print all inbook entires

\printbibsection{booklet}{invited talks \& organized workshops} % Print all booklet entries from the bibliography

\printbibsection{proceedings}{posters at international conferences} % Print all proceedings entries from the bibliography

\subsubsection{}{published code}
For an up-to-date list of published code projects, please visit my GitHub profile at
\href{https://github.com/gkiar}{https://github.com/gkiar}.
% \printbibsection{misc}{published code} % Print all miscellaneous entries from the bibliography

%----------------------------------------------------------------------------------------
\end{document}
